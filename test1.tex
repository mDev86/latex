% mathenv.tex - an example showing the different math environments, how
%               to use them and how they look.
%
% Andrew Roberts - 27th April 2004

\documentclass[a4paper]{article}

\usepackage{times}

% The mathptmx package does for maths equations what the times package
% does for the main text. That is, uses scalable fonts rather than the
% default bitmapped ones. This is useful if you later want to convert
% your postscript file to PDF.

\usepackage{mathptmx}

\begin{document}
\title{Using Maths Environments}
\author{Andrew Roberts}
\date{}
\maketitle

Maths is a pretty fundamental area with Latex, and with Tex! Normally,
environments require the standard \verb|\begin{...} \end{...}| format.
However, as it was assumed that maths stuff would be frequent in most
documents you created, then some short cuts were also added. You are
free to use the standard approach, but there are also two shortcuts: one
being the Latex way, and the other being the Tex way. 

\section{Text environments}

A text maths environment is one that displays the mathematical
equation/symbol inline, with the current text. This is opposed to the
displayed environments, as we shall see later, that separate the maths
from the text. So, if we wanted to mention that \(a + a = 2a\) within
current sentence, then this environment is the best to use.

\begin{table}[!htp]
\begin{tabular}{|l|l|l|}
\hline
  Method & Command & Output \\ \hline
  Standard & \verb|\begin{math}a + a = 2a\end{math}| & \begin{math}a + a = 2a\end{math} \\ \hline
  Latex & \verb|\(a + a = 2a\)| & \(a + a = 2a\) \\ \hline
  Tex & \verb|$a + a = 2a$| & $a + a = 2a$ \\ \hline
\end{tabular}
\caption{Ways to use the text maths enviroment}
\end{table}

\section{Displayed environments}

\subsection{\texttt{displaymath}}

Displayed maths environments position the maths within it differently.
It's separate from the preceeding text (and subsequent text for that
matter) as well as centered. It's also possible to number equations that
are in this mode. So, if we wanted to state that the equation to find
the volume of a sphere is: \[\frac{4}{3}\pi r^2\]

Even though in the Latex source file, the maths command is at the end of
the previous paragraph, Latex will not display it inline, as we can see.

\begin{table}[!htp]
\begin{tabular}{|l|l|l|}
\hline
  Method & Command \\ \hline
  Standard & \verb|\begin{displaymath}\frac{4}{3}\pi r^2\end{displaymath}| \\ \hline
  Latex & \verb|\[\frac{4}{3}\pi r^2\]| \\ \hline
  Tex & \verb|$$\frac{4}{3}\pi r^2$$| \\ \hline
\end{tabular}
\caption{Ways to use the displayed maths enviroment}
\end{table}

\subsection{\texttt{equation}}

The \texttt{equation} environment is almost identical to that of
\texttt{displaymath}, with the exception being that equation numbers are
also added along side the displayed mathematics. This is useful for
documents that contain many mathematical formulas and you wish to
cross-reference them in the main text. For example, see equation \ref{binomial} for
the formular for binomial distribution. Unfornately, there is no
shorthand way to enter the equation environment, just the usual
\verb|\begin{...} ... \end{...}|.

\begin{equation}
b\left(k; n, x\right) = \left( \begin{array}{c} n \\ k \end{array}\right) x^k\left(1-x\right)^{n-k}
\label{binomial}
\end{equation}


\end{document}
